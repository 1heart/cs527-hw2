\documentclass{article}

\newif\ifstays
\staysfalse                  % Change this to \staysfalse to make all sample text disappear

\usepackage{fullpage} % Makes the text margins smaller
\usepackage{graphicx} % To include figures
\usepackage{fancyvrb} % Includes the \VerbatimInput command to read in code files

\author{Yixin Lin, Cody Lieu}
\title{COMPSCI 527 Homework 2}

% A very simple environment for writing pseudo-code
\newenvironment{pgm}{
  \begin{center}\begin{tabbing}
  xx \= xx \= xx \= xx \= xx \= xx \= xx \= xx \= xx \= xx \= xx \= \kill\>\+}{
  \end{tabbing}\end{center}}

\begin{document}
\maketitle


%%% START OF TEXT TO REMOVE
\ifstays
\noindent [Please remove all the extra stuff below from the \verb#.tex# file before you hand in the resulting PDF file. However, please leave section headers and \verb#\newline# commands where they are. It is OK to add \verb#\newline# commands if you find that useful, but do so sparingly.

There are two ways to remove this extra stuff. One is to do so physically (look for matching \texttt{START/END}
comments), the other is to change the string \verb#\staystrue# close to the beginning of the file to \verb#\staysfalse#
\fi
%%% END OF TEXT TO REMOVE

\section*{Problem 1(a)}

\[
A_0 = \left[\begin{array}{*{4}c}
 1 & 0 & 1 \\
 0 & 1 & 0 \\
 1 & 1 & 0
 
 \end{array}\right]
\]


$$ \mathbf{q_1} = \frac{\mathbf{a_1}}{|\mathbf{a_1}|} =  \left[ \begin{array} {*{3}c} 
    \frac{1}{\sqrt{2}} \\
    0 \\
    \frac{1}{\sqrt{2}} \\
\end{array}\right]
$$

$$ \mathbf{q_2} = normalized(\mathbf{a_2} - proj_{\mathbf{q_1}}\mathbf{a2}) =  \left[ \begin{array} {*{3}c} 
    -\frac{\sqrt{6}}{6} \\
    \frac{\sqrt{6}}{3} \\
    \frac{\sqrt{6}}{6} \\
\end{array}\right]
$$

$$ \mathbf{q_3} = normalized(\mathbf{a_3} - proj_{\mathbf{q_1}}\mathbf{a3} -
proj_{\mathbf{q_2}}\mathbf{a3}) =  \left[ \begin{array} {*{3}c} 
    \frac{\sqrt{3}}{3} \\
    \frac{\sqrt{3}}{3} \\
    - \frac{\sqrt{3}}{3} \\
\end{array}\right]
$$

\[
Q = \left[\begin{array}{*{4}c}
    \frac{1}{\sqrt{2}} & -\frac{\sqrt{6}}{6} & \frac{\sqrt{3}}{3} \\
    0 & \frac{\sqrt{6}}{3} & \frac{\sqrt{3}}{3} \\
    \frac{1}{\sqrt{2}} & \frac{\sqrt{6}}{6} & - \frac{\sqrt{3}}{3}

 \end{array}\right]
\]


\section*{Problem 1(b)}

$$|\mathbf{q_1}| = \sqrt{(\frac{1}{\sqrt{2}})^2 +
    0^2 +
    (\frac{1}{\sqrt{2}})^2 } = 1$$

$$|\mathbf{q_2}| = \sqrt{(-\frac{\sqrt{6}}{6})^2 +
    (\frac{\sqrt{6}}{3})^2 +
    (\frac{\sqrt{6}}{6})^2 } = 1$$


$$|\mathbf{q_2}| = \sqrt{(\frac{\sqrt{3}}{3})^2 +
    (\frac{\sqrt{3}}{3})^2 +
    (-\frac{\sqrt{3}}{3} } = 1$$


$$
\mathbf{q_1} \cdot \mathbf{q_2} = 
\left[ \begin{array} {*{3}c} 
    \frac{1}{\sqrt{2}} \\
    0 \\
    \frac{1}{\sqrt{2}} \\
\end{array}\right] 
\cdot 
\left[ \begin{array} {*{3}c} 
    \frac{\sqrt{3}}{3} \\
    \frac{\sqrt{3}}{3} \\
    - \frac{\sqrt{3}}{3} \\
\end{array}\right]
= \mathbf{0}
$$

$$
\mathbf{q_2} \cdot \mathbf{q_3} = 
\left[ \begin{array} {*{3}c} 
    -\frac{\sqrt{6}}{6} \\
    \frac{\sqrt{6}}{3} \\
    \frac{\sqrt{6}}{6} \\
\end{array}\right] 
\cdot 
\left[ \begin{array} {*{3}c} 
    \frac{\sqrt{3}}{3} \\
    \frac{\sqrt{3}}{3} \\
    - \frac{\sqrt{3}}{3} \\
\end{array}\right]
= \mathbf{0}
$$

$$
\mathbf{q_1} \cdot \mathbf{q_3} = 
\left[ \begin{array} {*{3}c} 
    \frac{1}{\sqrt{2}} \\
    0 \\
    \frac{1}{\sqrt{2}} \\
\end{array}\right] 
\cdot 
\left[ \begin{array} {*{3}c} 
    \frac{\sqrt{3}}{3} \\
    \frac{\sqrt{3}}{3} \\
    - \frac{\sqrt{3}}{3} \\
\end{array}\right]
= \mathbf{0}
$$

\section*{Problem 1(c)}

$r$ is equal to the rank of the matrix $A$.

\section*{Problem 1(d)}

Yes, since Gram-Schmidt gives us an orthogonal basis for the column space, it gives us the dimension of the column space, which is equal to the rank.

\section*{Problem 1(e)}

\section*{Problem 1(f)}


$$ r_ij = \mathbf{q_i} \cdot \mathbf{a_j} $$

$$ r_jj = |\mathbf{a_j'}|$$

\section*{Problem 1(g)}

\[
R = \left[\begin{array}{*{4}c}
 \ast & \ast  & \ast & \ast\\
 & \ast & \ast & \ast\\
 &  & \ast &\ast \\
 &  &  & \ast
\end{array}\right]
\]

%%% START OF TEXT TO REMOVE
\ifstays
Here is how you would write the fill pattern for a $4\times 4$ identity matrix:
\[
I = \left[\begin{array}{*{4}c}
 \ast &  &  & \\
 & \ast &  & \\
 &  & \ast & \\
 &  &  & \ast
\end{array}\right]
\]
\fi
%%% END OF TEXT TO REMOVE

\section*{Problem 1(h)}

\[
Q = \left[\begin{array}{*{4}c}
 \mathbf{q_1} & \mathbf{q_2}  & \mathbf{q_3}\\
\end{array}\right]
\]

\[
R = \left[\begin{array}{*{4}c}
 \ast & \ast  & \ast & \ast\\
 & \ast & \ast & \ast\\
 &  &  &\ast  \\
\end{array}\right]
\]

In the third iteration of the \verb#for# loop, the \verb#if# statement fails and therefore $r$ is not incremented. The total number of columns in $q$ is $r$, so there is one less column in $Q$. This makes sense, since $Q$ should be an orthogonal matrix whose column space is the same as $A$; if $A$ has linearly dependent vectors, then the number of column vectors in $Q$ will be reduced.

$r_{jj} = |\mathbf{a_j'}| = 0$, so the last element in the diagonal will be 0.


\section*{Problem 1(i)}

$$Q_{m \times r}$$
$$R_{r \times n}$$

\section*{Problem 1(j)}

\section*{Problem 1(k)}

\section*{Problem 2(a)}

$$ \mathbf{c} = Q^{-1}\mathbf{b}$$

\section*{Problem 2(b)}

Since $Q$ is an orthogonal matrix, $Q^{-1} = Q^{T}$. Therefore, we don't require the expensive computation of determining the inverse of $Q$ and can instead take the transpose (which is very efficient).

\section*{Problem 2(c)}

Since $R$ is a triangular matrix, apply the algorithm of backward substitution:

%%% START OF TEXT TO REMOVE
\ifstays
Here is one way to render the Gram-Schmidt pseudo-code in \LaTeX. You can use this as a template to write your own pseudo-code.
\newcommand{\ba}{\mathbf{a}}
\newcommand{\bq}{\mathbf{q}}
\begin{pgm}
\mbox{}\+\+\+\+\\
$r = 0$\\
for $j=1$ to $n$\+\\
$\ba'_{j} = \ba_{j} - \sum_{i=1}^{r} (\bq_{i}^{T}\ba_{j})\bq_{i}$\\
if $\|\ba'_{j}\| \neq \mathbf{0}$\+\\
  $r = r+1$\\
  $\bq_{r} = \frac{\ba'_{j}}{\|\ba'_{j}\|}$\-\\
end\-\\
end
\end{pgm}
The \verb#\+# and \verb#\-# commands tell the interpreter respectively to add or remove one indentation tab from subsequent lines. The \texttt{pgm} environment is defined for you in the preamble of the \verb#template.tex# file.
\fi
%%% END OF TEXT TO REMOVE

\section*{Problem 2(d)}

Since some columns of A are linearly independent, the solution has free variables. In order to pick one solution, pick a free variable at random and then back-sbustitute:

\section*{Problem 2(e)}

\section*{Problem 2(f)}

\section*{Problem 2(g)}

\section*{Problem 2(h)}

\section*{Problem 2(i)}

See next page.

% \input{tests} % Uncomment this line to include your test results, which will show up on a fresh page


\end{document}
