\documentclass{article}

\newcommand{\itab}[1]{\hspace{0em}\rlap{#1}}
\newcommand{\tab}[1]{\hspace{.05\textwidth}\rlap{#1}}


\newif\ifstays
\staysfalse                  % Change this to \staysfalse to make all sample text disappear

\usepackage{fullpage} % Makes the text margins smaller
\usepackage{graphicx} % To include figures
\usepackage{fancyvrb} % Includes the \VerbatimInput command to read in code files

\usepackage{url}
\usepackage{hyperref}

\author{Yixin Lin, Cody Lieu}
\title{COMPSCI 527 Homework 2}

% A very simple environment for writing pseudo-code
\newenvironment{pgm}{
  \begin{center}\begin{tabbing}
  xx \= xx \= xx \= xx \= xx \= xx \= xx \= xx \= xx \= xx \= xx \= \kill\>\+}{
  \end{tabbing}\end{center}}

\begin{document}
\maketitle


%%% START OF TEXT TO REMOVE
\ifstays
\noindent [Please remove all the extra stuff below from the \verb#.tex# file before you hand in the resulting PDF file. However, please leave section headers and \verb#\newline# commands where they are. It is OK to add \verb#\newline# commands if you find that useful, but do so sparingly.

There are two ways to remove this extra stuff. One is to do so physically (look for matching \texttt{START/END}
comments), the other is to change the string \verb#\staystrue# close to the beginning of the file to \verb#\staysfalse#
\fi
%%% END OF TEXT TO REMOVE

\section*{Problem 1(a)}

\[
A_0 = \left[\begin{array}{*{4}c}
 1 & 0 & 1 \\
 0 & 1 & 0 \\
 1 & 1 & 0
 
 \end{array}\right]
\]


$$ \mathbf{q_1} = \frac{\mathbf{a_1}}{|\mathbf{a_1}|} =  \left[ \begin{array} {*{3}c} 
    \frac{1}{\sqrt{2}} \\
    0 \\
    \frac{1}{\sqrt{2}} \\
\end{array}\right]
$$

$$ \mathbf{q_2} = normalized(\mathbf{a_2} - proj_{\mathbf{q_1}}\mathbf{a2}) =  \left[ \begin{array} {*{3}c} 
    -\frac{\sqrt{6}}{6} \\
    \frac{\sqrt{6}}{3} \\
    \frac{\sqrt{6}}{6} \\
\end{array}\right]
$$

$$ \mathbf{q_3} = normalized(\mathbf{a_3} - proj_{\mathbf{q_1}}\mathbf{a3} -
proj_{\mathbf{q_2}}\mathbf{a3}) =  \left[ \begin{array} {*{3}c} 
    \frac{\sqrt{3}}{3} \\
    \frac{\sqrt{3}}{3} \\
    - \frac{\sqrt{3}}{3} \\
\end{array}\right]
$$

\[
Q = \left[\begin{array}{*{4}c}
    \frac{1}{\sqrt{2}} & -\frac{\sqrt{6}}{6} & \frac{\sqrt{3}}{3} \\
    0 & \frac{\sqrt{6}}{3} & \frac{\sqrt{3}}{3} \\
    \frac{1}{\sqrt{2}} & \frac{\sqrt{6}}{6} & - \frac{\sqrt{3}}{3}

 \end{array}\right]
\]


\section*{Problem 1(b)}

$$|\mathbf{q_1}| = \sqrt{(\frac{1}{\sqrt{2}})^2 +
    0^2 +
    (\frac{1}{\sqrt{2}})^2 } = 1$$

$$|\mathbf{q_2}| = \sqrt{(-\frac{\sqrt{6}}{6})^2 +
    (\frac{\sqrt{6}}{3})^2 +
    (\frac{\sqrt{6}}{6})^2 } = 1$$


$$|\mathbf{q_2}| = \sqrt{(\frac{\sqrt{3}}{3})^2 +
    (\frac{\sqrt{3}}{3})^2 +
    (-\frac{\sqrt{3}}{3} } = 1$$


$$
\mathbf{q_1} \cdot \mathbf{q_2} = 
\left[ \begin{array} {*{3}c} 
    \frac{1}{\sqrt{2}} \\
    0 \\
    \frac{1}{\sqrt{2}} \\
\end{array}\right] 
\cdot 
\left[ \begin{array} {*{3}c} 
    \frac{\sqrt{3}}{3} \\
    \frac{\sqrt{3}}{3} \\
    - \frac{\sqrt{3}}{3} \\
\end{array}\right]
= \mathbf{0}
$$

$$
\mathbf{q_2} \cdot \mathbf{q_3} = 
\left[ \begin{array} {*{3}c} 
    -\frac{\sqrt{6}}{6} \\
    \frac{\sqrt{6}}{3} \\
    \frac{\sqrt{6}}{6} \\
\end{array}\right] 
\cdot 
\left[ \begin{array} {*{3}c} 
    \frac{\sqrt{3}}{3} \\
    \frac{\sqrt{3}}{3} \\
    - \frac{\sqrt{3}}{3} \\
\end{array}\right]
= \mathbf{0}
$$

$$
\mathbf{q_1} \cdot \mathbf{q_3} = 
\left[ \begin{array} {*{3}c} 
    \frac{1}{\sqrt{2}} \\
    0 \\
    \frac{1}{\sqrt{2}} \\
\end{array}\right] 
\cdot 
\left[ \begin{array} {*{3}c} 
    \frac{\sqrt{3}}{3} \\
    \frac{\sqrt{3}}{3} \\
    - \frac{\sqrt{3}}{3} \\
\end{array}\right]
= \mathbf{0}
$$

\section*{Problem 1(c)}

$r$ is equal to the rank of the matrix $A$.

\section*{Problem 1(d)}

Yes, since Gram-Schmidt gives us an orthogonal basis for the column space when applied to the column vectors in $A$, it gives us the dimension of the column space, which is equal to the rank.

\section*{Problem 1(e)}

First, apply Gram-Schmidt on the set of column vectors of A; then add $\mathbf{b}$ to the set and continue. If adding $\mathbf{b}$ increased the number of orthogonormal vectors, that means there is no solution (i.e. there is no linear combination of basis vectors of the column space of A that equals $\mathbf{b}$). Otherwise, $\mathbf{b}$ is a linear combination of the column vectors of A, and there exists a solution.

% $Q$ forms a basis for the vector space $A$. This means that if a solution exists to $A\mathbf{x}=\mathbf{b}$, then $\mathbf{b}$ can be expressed as a linear combination of the orthonormal vectors in $Q$, i.e. if a solution exists to the linear system $Q\mathbf{x'}=\mathbf{b}$, where $\mathbf{x'}$ is a r-by-1 vector of constants (not necessarily satisfying $A\mathbf{x}=\mathbf{b}$), then a solution exists to $A\mathbf{x}=\mathbf{b}$.

\section*{Problem 1(f)}


$$ r_{ij} = \mathbf{q_i} \cdot \mathbf{a_j} $$

$$ r_{jj} = |\mathbf{a_j'}|$$

\section*{Problem 1(g)}

\[
R = \left[\begin{array}{*{4}c}
 \ast & \ast  & \ast & \ast\\
 & \ast & \ast & \ast\\
 &  & \ast &\ast \\
 &  &  & \ast
\end{array}\right]
\]

%%% START OF TEXT TO REMOVE
\ifstays
Here is how you would write the fill pattern for a $4\times 4$ identity matrix:
\[
I = \left[\begin{array}{*{4}c}
 \ast &  &  & \\
 & \ast &  & \\
 &  & \ast & \\
 &  &  & \ast
\end{array}\right]
\]
\fi
%%% END OF TEXT TO REMOVE

\section*{Problem 1(h)}

\[
Q = \left[\begin{array}{*{4}c}
 \mathbf{q_1} & \mathbf{q_2}  & \mathbf{q_3}\\
\end{array}\right]
\]

\[
R = \left[\begin{array}{*{4}c}
 \ast & \ast  & \ast & \ast\\
 & \ast & \ast & \ast\\
 &  &  &\ast  \\
\end{array}\right]
\]

In the third iteration of the \verb#for# loop, the \verb#if# statement fails and therefore $r$ is not incremented. The total number of columns in $q$ is $r$, so there is one less column in $Q$. This makes sense, since $Q$ should be an orthogonal matrix whose column space is the same as $A$; if $A$ has linearly dependent vectors, then the number of column vectors in $Q$ will be reduced.

$r_{jj} = |\mathbf{a_j'}| = 0$, so the last element in the diagonal will be 0.

\section*{Problem 1(i)}

$$Q_{m \times r}$$
$$R_{r \times n}$$

\section*{Problem 1(j)}

\begin{verbatim}
function [Q, R] = gs(A)

Q = [];
R = [];

r = 0;
for j = 1:size(A, 2)
    ap = A(:, j);
    for i = 1:r
        R(i, j) = dot(Q(:, i), A(:, j));
        ap = ap - R(i, j).*Q(:, i);
    end
    rjj = norm(ap);
    if rjj > sqrt(eps)
        r = r + 1;
        R(j,j) = rjj;
        Q(:, r) = ap/rjj;
    end
end
\end{verbatim}

\section*{Problem 1(k)}

\begin{verbatim}
function [Q, R] = ggs(A, Q, R)

if nargin < 3 || isempty(Q) || isempty(R)
    Q = [];
    R = [];
end

[m, n] = size(A);
[r0, n0] = size(R);

if ~isempty(Q) && size(Q, 1) ~= m
    error('A and Q have inconsistent row sizes')
end

if ~isempty(Q) && size(Q, 2) ~= r0
    error('Q and R have inconsistent sizes')
end

[qr, qc] = size(Q);

% Your code here
r = qc;
for j = 1+qc:n+qc
    ap = A(:, j-qc);
    for i = 1:r
        R(i, j) = dot(Q(:, i), A(:, j-qc));
        ap = ap - R(i, j).*Q(:, i);
    end
    rjj = norm(ap);
    if rjj > sqrt(eps)
        r = r + 1;
        R(j,j) = rjj;
        Q(:, r) = ap/rjj;
    end
end
\end{verbatim}

\section*{Problem 2(a)}

$$ \mathbf{c} = Q^{-1}\mathbf{b} = Q^{T}\mathbf{b}$$

\section*{Problem 2(b)}

Since $Q$ is an orthogonal matrix, $Q^{-1} = Q^{T}$. Therefore, we don't require the expensive computation of determining the inverse of $Q$ and can instead take the transpose (which is very efficient).

\section*{Problem 2(c)}

Since $R$ is a triangular matrix, apply the algorithm of backward substitution pseudo-coded below:\\\\
x = new arr[n]\\
for i = n to 1

\itab{x[i] = c[i]}

\itab{for j = i + 1 to n}

\tab{x[i] = x[i] - x[j] * R[i][j]}

\itab{end}

\itab{x[i] = x[i] / R[i][i]}\\
end

%%% START OF TEXT TO REMOVE
\ifstays
Here is one way to render the Gram-Schmidt pseudo-code in \LaTeX. You can use this as a template to write your own pseudo-code.
\newcommand{\ba}{\mathbf{a}}
\newcommand{\bq}{\mathbf{q}}
\begin{pgm}
\mbox{}\+\+\+\+\\
$r = 0$\\
for $j=1$ to $n$\+\\
$\ba'_{j} = \ba_{j} - \sum_{i=1}^{r} (\bq_{i}^{T}\ba_{j})\bq_{i}$\\
if $\|\ba'_{j}\| \neq \mathbf{0}$\+\\
  $r = r+1$\\
  $\bq_{r} = \frac{\ba'_{j}}{\|\ba'_{j}\|}$\-\\
end\-\\
end
\end{pgm}
The \verb#\+# and \verb#\-# commands tell the interpreter respectively to add or remove one indentation tab from subsequent lines. The \texttt{pgm} environment is defined for you in the preamble of the \verb#template.tex# file.
\fi
%%% END OF TEXT TO REMOVE

\section*{Problem 2(d)}

Since some columns of A are linearly independent, the solution has free variables. In order to pick one solution, pick a free variable at random and then back-substitute:

\section*{Problem 2(e)}

Since $leftnull(A) = range(A)^{\perp}$, we can find a basis of $R^m$ by using the identity matrix as A and keeping Q and R. Then we remove the vectors that were already in Q to find a basis for the orthogonal complement of $range(A)$.

\begin{verbatim}
[m, n] = size(Q);
[Qn, Rn] = ggs(eye(m), Q, R);
L = Qn(:, n+1:end);
\end{verbatim}

\section*{Problem 2(f)}

% Q and R come from ggs; L come from 2e. N and W come from ggs of $A^T$. $x$ can be gotten from 2(a-d).
% Are there better ways to get N and W?

\section*{Problem 2(g)}

% Rank < n
The system in eq. 3 admits infinitely many solutions if the rank of A is less than the number of columns in A (i.e. \verb#size(Q,2) < size(A,2)#) and x exists (i.e. \verb#x~=[]#).

The set of solutions is $ \mathbf{n} + \mathbf{x}$, $ \forall n \in N $. This is the set of vectors in $N$, shifted by $\mathbf{x}$, which is an \textbf{affine space}.

This is true because of the following theorem, taken verbatim from \href{https://www.math.stonybrook.edu/~badger/mat211f12/solver2.pdf}{this link from Stony Brook}:

\subsubsection*{Theorem}

Let $x_p$ be a $particular $ $solution$ to $Ax = b$. Then the $general$ $ solution$ to $Ax = b$ is given by $x_p + x_n$ where $x_n$ is a vector in the nullspace of $A$.


% In this case, the set of solutions to the system $R\mathbf{x}=\mathbf{c}$ can be defined precisely by the space spanned by null(A). null(A) is a vector space.

\section*{Problem 2(h)}

% Check that every inner product between a pair of vectors is 0

Check that $\mathbf{n} \cdot \mathbf{w} = \mathbf{0}$ $ \forall \mathbf{n} \in N$, $ \forall \mathbf{w} \in W$. Equivalently, \verb#~any(N.' * W)  # (Replace N and W with L and Q respectively, to check for L and Q.)

\section*{Problem 2(i)}

See next page.

\newpage
\footnotesize{
Test case 1: A has rank 1. Product checks passed. Dimension checks passed.
\[
\begin{array}{*{4}{c}}
A = \left[\begin{array}{*{2}c}
	-0.32 & -0.05\\
	-0.55 & -0.08
\end{array}\right]
\;, & 
\mathbf{b} = \left[\begin{array}{*{1}c}
	-1.48\\
	0.54
\end{array}\right]
\;, & 
\mathbf{x} = []\;, & 
R = \left[\begin{array}{*{2}c}
	0.63 & 0.10
\end{array}\right]
\\
\\\mbox{}\\
N = \left[\begin{array}{*{1}c}
	0.15\\
	-0.99
\end{array}\right]
\;, & 
W = \left[\begin{array}{*{1}c}
	-0.99\\
	-0.15
\end{array}\right]
\;, & 
L = \left[\begin{array}{*{1}c}
	0.86\\
	-0.50
\end{array}\right]
\;, & 
Q = \left[\begin{array}{*{1}c}
	-0.50\\
	-0.86
\end{array}\right]
\\
\end{array}
\]
\hrulefill

Test case 2: A has rank 2. Product checks passed. Dimension checks passed.
\[
\begin{array}{*{4}{c}}
A = \left[\begin{array}{*{2}c}
	-0.93 & -0.56\\
	-0.53 & -0.59
\end{array}\right]
\;, & 
\mathbf{b} = \left[\begin{array}{*{1}c}
	-1.00\\
	0.15
\end{array}\right]
\;, & 
\mathbf{x} = \left[\begin{array}{*{1}c}
	2.66\\
	-2.66
\end{array}\right]
\;, & 
R = \left[\begin{array}{*{2}c}
	1.07 & 0.78\\
	0.00 & 0.24
\end{array}\right]
\\
\\\mbox{}\\
N = []\;, & 
W = \left[\begin{array}{*{2}c}
	-0.86 & 0.51\\
	-0.51 & -0.86
\end{array}\right]
\;, & 
L = []\;, & 
Q = \left[\begin{array}{*{2}c}
	-0.87 & 0.50\\
	-0.50 & -0.87
\end{array}\right]
\\
\end{array}
\]
\hrulefill

Test case 3: A has rank 2. Product checks passed. Dimension checks passed.
\[
\begin{array}{*{4}{c}}
A = \left[\begin{array}{*{2}c}
	0.41 & 0.55\\
	-0.34 & -0.19\\
	-0.55 & -0.44
\end{array}\right]
\;, & 
\mathbf{b} = \left[\begin{array}{*{1}c}
	0.23\\
	-0.05\\
	-0.15
\end{array}\right]
\;, & 
\mathbf{x} = \left[\begin{array}{*{1}c}
	-0.15\\
	0.53
\end{array}\right]
\;, & 
R = \left[\begin{array}{*{2}c}
	0.76 & 0.69\\
	0.00 & 0.22
\end{array}\right]
\\
\\\mbox{}\\
N = []\;, & 
W = \left[\begin{array}{*{2}c}
	0.60 & -0.80\\
	0.80 & 0.60
\end{array}\right]
\;, & 
L = \left[\begin{array}{*{1}c}
	0.25\\
	-0.73\\
	0.64
\end{array}\right]
\;, & 
Q = \left[\begin{array}{*{2}c}
	0.54 & 0.80\\
	-0.44 & 0.53\\
	-0.72 & 0.28
\end{array}\right]
\\
\end{array}
\]
\hrulefill

Test case 4: A has rank 1. Product checks passed. Dimension checks passed.
\[
\begin{array}{*{4}{c}}
A = \left[\begin{array}{*{2}c}
	-0.30 & -0.31\\
	-0.23 & -0.24\\
	-0.67 & -0.69
\end{array}\right]
\;, & 
\mathbf{b} = \left[\begin{array}{*{1}c}
	-0.07\\
	-0.06\\
	-0.16
\end{array}\right]
\;, & 
\mathbf{x} = \left[\begin{array}{*{1}c}
	0.12\\
	0.12
\end{array}\right]
\;, & 
R = \left[\begin{array}{*{2}c}
	0.77 & 0.80
\end{array}\right]
\\
\\\mbox{}\\
N = \left[\begin{array}{*{1}c}
	0.72\\
	-0.69
\end{array}\right]
\;, & 
W = \left[\begin{array}{*{1}c}
	-0.69\\
	-0.72
\end{array}\right]
\;, & 
L = \left[\begin{array}{*{2}c}
	0.92 & 0.00\\
	-0.13 & 0.95\\
	-0.37 & -0.32
\end{array}\right]
\;, & 
Q = \left[\begin{array}{*{1}c}
	-0.39\\
	-0.30\\
	-0.87
\end{array}\right]
\\
\end{array}
\]
\hrulefill

Test case 5: A has rank 1. Product checks passed. Dimension checks passed.
\[
\begin{array}{*{4}{c}}
A = \left[\begin{array}{*{3}c}
	-0.52 & -0.19 & -0.47\\
	-0.54 & -0.20 & -0.49
\end{array}\right]
\;, & 
\mathbf{b} = \left[\begin{array}{*{1}c}
	-1.29\\
	0.34
\end{array}\right]
\;, & 
\mathbf{x} = []\;, & 
R = \left[\begin{array}{*{3}c}
	0.75 & 0.28 & 0.68
\end{array}\right]
\\
\\\mbox{}\\
N = \left[\begin{array}{*{2}c}
	0.70 & 0.00\\
	-0.27 & 0.93\\
	-0.66 & -0.38
\end{array}\right]
\;, & 
W = \left[\begin{array}{*{1}c}
	-0.72\\
	-0.26\\
	-0.65
\end{array}\right]
\;, & 
L = \left[\begin{array}{*{1}c}
	0.72\\
	-0.70
\end{array}\right]
\;, & 
Q = \left[\begin{array}{*{1}c}
	-0.70\\
	-0.72
\end{array}\right]
\\
\end{array}
\]
\hrulefill

Test case 6: A has rank 2. Product checks passed. Dimension checks passed.
\[
\begin{array}{*{4}{c}}
A = \left[\begin{array}{*{3}c}
	-0.62 & -0.79 & -0.77\\
	-0.06 & -0.11 & 0.40
\end{array}\right]
\;, & 
\mathbf{b} = \left[\begin{array}{*{1}c}
	-1.23\\
	0.46
\end{array}\right]
\;, & 
\mathbf{x} = \left[\begin{array}{*{1}c}
	0.20\\
	0.19\\
	1.24
\end{array}\right]
\;, & 
R = \left[\begin{array}{*{3}c}
	0.63 & 0.79 & 0.72\\
	0.00 & 0.03 & -0.47
\end{array}\right]
\\
\\\mbox{}\\
N = \left[\begin{array}{*{1}c}
	0.80\\
	-0.60\\
	-0.04
\end{array}\right]
\;, & 
W = \left[\begin{array}{*{2}c}
	-0.49 & -0.34\\
	-0.62 & -0.50\\
	-0.61 & 0.79
\end{array}\right]
\;, & 
L = []\;, & 
Q = \left[\begin{array}{*{2}c}
	-0.99 & 0.10\\
	-0.10 & -0.99
\end{array}\right]
\\
\end{array}
\]
\hrulefill

} % Uncomment this line to include your test results, which will show up on a fresh page


\end{document}
